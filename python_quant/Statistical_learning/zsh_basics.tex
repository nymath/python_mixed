% Options for packages loaded elsewhere
\PassOptionsToPackage{unicode}{hyperref}
\PassOptionsToPackage{hyphens}{url}
%
\documentclass[
]{article}
\usepackage{amsmath,amssymb}
\usepackage{lmodern}
\usepackage{iftex}
\ifPDFTeX
  \usepackage[T1]{fontenc}
  \usepackage[utf8]{inputenc}
  \usepackage{textcomp} % provide euro and other symbols
\else % if luatex or xetex
  \usepackage{unicode-math}
  \defaultfontfeatures{Scale=MatchLowercase}
  \defaultfontfeatures[\rmfamily]{Ligatures=TeX,Scale=1}
\fi
% Use upquote if available, for straight quotes in verbatim environments
\IfFileExists{upquote.sty}{\usepackage{upquote}}{}
\IfFileExists{microtype.sty}{% use microtype if available
  \usepackage[]{microtype}
  \UseMicrotypeSet[protrusion]{basicmath} % disable protrusion for tt fonts
}{}
\makeatletter
\@ifundefined{KOMAClassName}{% if non-KOMA class
  \IfFileExists{parskip.sty}{%
    \usepackage{parskip}
  }{% else
    \setlength{\parindent}{0pt}
    \setlength{\parskip}{6pt plus 2pt minus 1pt}}
}{% if KOMA class
  \KOMAoptions{parskip=half}}
\makeatother
\usepackage{xcolor}
\usepackage{listings}
\newcommand{\passthrough}[1]{#1}
\lstset{defaultdialect=[5.3]Lua}
\lstset{defaultdialect=[x86masm]Assembler}
\setlength{\emergencystretch}{3em} % prevent overfull lines
\providecommand{\tightlist}{%
  \setlength{\itemsep}{0pt}\setlength{\parskip}{0pt}}
\setcounter{secnumdepth}{-\maxdimen} % remove section numbering
\ifLuaTeX
  \usepackage{selnolig}  % disable illegal ligatures
\fi
\IfFileExists{bookmark.sty}{\usepackage{bookmark}}{\usepackage{hyperref}}
\IfFileExists{xurl.sty}{\usepackage{xurl}}{} % add URL line breaks if available
\urlstyle{same} % disable monospaced font for URLs
\hypersetup{
  hidelinks,
  pdfcreator={LaTeX via pandoc}}

\author{}
\date{}

\begin{document}

\leavevmode\vadjust pre{\hypertarget{3964f9c8}{}}%
本节尝试使用在jupyter
notebook使用zsh,顺便写一个zsh的食用指南,notebook写真的方便很多啊。

\hypertarget{2f391477}{}
\begin{lstlisting}[language=Python]
import zsh_in_jupyter
\end{lstlisting}

\hypertarget{5198530e}{}
\begin{lstlisting}[language=Python]
\end{lstlisting}

\hypertarget{426de12e}{}
\hypertarget{introdution}{%
\section{Introdution}\label{introdution}}

因为及其好用,比如 具体而言 \textbackslash begin\{itemize\}
\textbackslash item 许多功能在图形界面不提供,只有通过命令行来实现。
\textbackslash item Finder 会隐藏许多你不太会需要的文件,而 Command line
则可以访问所有文件。 \textbackslash item Command line 可以通过 SSH
远程访问你的 Mac 或者 Linux。 \textbackslash item 普通用户可以通过
\passthrough{\lstinline!sudo!} 命令获得 root 用户权限。
\textbackslash item
如果你开启手动输入用户名登陆模式,登陆时在用户名处输入
\passthrough{\lstinline!>console!}
可以直接进入命令行界面。随后你仍然需要登录到一个账户。
\textbackslash end\{itemize\}

\hypertarget{37695442}{}
\begin{lstlisting}[language=Python]
!echo ~
\end{lstlisting}

\begin{lstlisting}
/Users/nymath
\end{lstlisting}

\hypertarget{a302ba1e}{}
\begin{lstlisting}[language=Python]
get_ipython().system('echo ~')
\end{lstlisting}

\begin{lstlisting}
/Users/nymath
\end{lstlisting}

\hypertarget{a524a580}{}
\begin{lstlisting}[language=Python]
# 打开当前文件夹
!temp=$(pwd);\
open $temp
\end{lstlisting}

\hypertarget{4831e23d}{}
\begin{lstlisting}[language=Python]
def zsh(par='temp=$(pwd);open $temp'):
    if type(par) == str:
        return get_ipython().system(par)
    else:
        return 'type error'
\end{lstlisting}

\hypertarget{ca8dc0b8}{}
\begin{lstlisting}[language=Python]
func.system('temp=$(pwd);open $temp')
\end{lstlisting}

\hypertarget{f017b2d9}{}
\begin{lstlisting}[language=Python]
get_ipython().system('temp=$(pwd);open $temp')
\end{lstlisting}

\hypertarget{ce7fec45}{}
\begin{lstlisting}[language=Python]
from zsh_in_jupyter import 
\end{lstlisting}

\hypertarget{d9c1184a}{}
\hypertarget{terminlogy}{%
\subsection{Terminlogy}\label{terminlogy}}

\begin{itemize}
\tightlist
\item
  \textbf{Command line}: A command-line interface is a way of giving
  instructions to a computer and getting results back
\item
  \textbf{Shell}: A shell is a program that creates a user interface of
  one kind or another enabling you to interact with a computer

  \begin{itemize}
  \tightlist
  \item
    \textbf{csh}: the C shell, named for similarities to the C
    programming language
  \item
    \textbf{zsh}: the Z shell, an advanced shell named after Yale
    professor Zhong Shao that incorporates features from tcsh, ksh, and
    bash, plus other capabilities
  \item
    \textbf{dash}: the Debian Almquist shell, a lightweight shell that's
    been around for more than two decades, but was not included with
    macOS until Catalina
  \end{itemize}
\item
  \textbf{Terminal}: the devices people used to interact with computers
  back in the days of monolithic mainframes.
\end{itemize}

\hypertarget{1f3f48b1}{}

\hypertarget{8cde7b83}{}
\hypertarget{commands-aruguments-and-flags}{%
\subsection{commands, aruguments and
flags}\label{commands-aruguments-and-flags}}

\begin{itemize}
\tightlist
\item
  \textbf{Command}: Commands are straightforward; they're the verbs of
  the command line (eventhough they may look nothing like English
  verbs).
\end{itemize}

\begin{lstlisting}
pwd
date
\end{lstlisting}

\begin{itemize}
\tightlist
\item
  \textbf{Arguments}: Along with commands (verbs), we have arguments,
  which you can think ofas nouns
\end{itemize}

比如你要处理一个文件,需要指定这个文件的名字

\begin{lstlisting}
nano file1
\end{lstlisting}

\begin{itemize}
\tightlist
\item
  \textbf{Flags}: Besides verbs and nouns, we have adverbs! In English,
  I could say, ``Eatcereal quickly!'' or ``Watch TV quietly.''
\end{itemize}

ls展示文件,ls a展示所有文件

\hypertarget{4d64b54c}{}
\begin{lstlisting}[language=Python]
!pwd; \
date
\end{lstlisting}

\begin{lstlisting}
/Users/nymath/Desktop/python/python_quant/Statistical_learning
Fri Sep 30 00:41:43 CST 2022
\end{lstlisting}

\hypertarget{ecc4840a}{}
\begin{lstlisting}[language=Python]
!ls -l -a
\end{lstlisting}

\begin{lstlisting}
total 88
drwxrwxr-x@  7 nymath  staff    224 Sep 30 00:49 .
drwxr-xr-x@ 16 nymath  staff    512 Sep 30 00:35 ..
drwxr-xr-x   4 nymath  staff    128 Sep 29 21:14 .ipynb_checkpoints
-rw-r--r--   1 nymath  staff  22692 Sep 30 00:22 Structures of Data.ipynb
-rw-r--r--@  1 nymath  staff    158 Sep 30 00:22 fib.py
-rw-------   1 nymath  staff    170 Sep 30 00:12 fib.py.save
-rw-r--r--   1 nymath  staff  11053 Sep 30 00:49 zsh_basics.ipynb
\end{lstlisting}

\hypertarget{0e93f1a0}{}
\begin{lstlisting}[language=Python]
!ls -a; #
\end{lstlisting}

\begin{lstlisting}
.                        Structures of Data.ipynb zsh_basics.ipynb
..                       fib.py
.ipynb_checkpoints       fib.py.save
\end{lstlisting}

\hypertarget{2b2a76af}{}
\begin{lstlisting}[language=Python]
!ls -a f* # show you only the names of files beginning with the letter f
\end{lstlisting}

\begin{lstlisting}
fib.py      fib.py.save
\end{lstlisting}

\hypertarget{84684b9b}{}
\begin{lstlisting}[language=Python]
\end{lstlisting}

\hypertarget{af360aad}{}
\hypertarget{examples}{%
\section{Examples}\label{examples}}

\hypertarget{9de924ca}{}
\hypertarget{open}{%
\subsection{open}\label{open}}

open
命令用于打开一个文件夹或者具体的文件,如果指定的是applications中的.app文件,则可以直接打开。

\hypertarget{174159de}{}
\begin{lstlisting}[language=Python]
!open /Applications/Notability.app 
\end{lstlisting}

\hypertarget{025763cf}{}
\begin{lstlisting}[language=Python]
!open /Applications/Safari.app
\end{lstlisting}

\hypertarget{81284268}{}
\begin{lstlisting}[language=Python]
!open /Applications/WeChat.app
\end{lstlisting}

\hypertarget{cc31e69f}{}
\begin{lstlisting}[language=Python]
!temp=$(pwd);\
open $temp
\end{lstlisting}

\hypertarget{144151c5}{}
\hypertarget{find--which}{%
\subsection{find \& which}\label{find--which}}

We can type the command "find" to find the files whose name contains the
name given by input. To be specific, if we want to find site-packages in
/opt/, we can

\hypertarget{e45b3c74}{}
\begin{lstlisting}[language=Python]
!find /opt/anaconda3/lib/python3.9 -name "*site-packages*"
\end{lstlisting}

\begin{lstlisting}
/opt/anaconda3/lib/python3.9/site-packages
\end{lstlisting}

\hypertarget{8350374f}{}
\begin{lstlisting}[language=Python]
!find /opt/anaconda3 -name "nbconvert"
\end{lstlisting}

\begin{lstlisting}
/opt/anaconda3/pkgs/jupyter_server-1.13.5-pyhd3eb1b0_0/site-packages/jupyter_server/tests/nbconvert
/opt/anaconda3/pkgs/jupyter_server-1.13.5-pyhd3eb1b0_0/site-packages/jupyter_server/tests/services/nbconvert
/opt/anaconda3/pkgs/jupyter_server-1.13.5-pyhd3eb1b0_0/site-packages/jupyter_server/nbconvert
/opt/anaconda3/pkgs/jupyter_server-1.13.5-pyhd3eb1b0_0/site-packages/jupyter_server/services/nbconvert
/opt/anaconda3/pkgs/notebook-6.4.8-py39hecd8cb5_0/lib/python3.9/site-packages/notebook/nbconvert
/opt/anaconda3/pkgs/notebook-6.4.8-py39hecd8cb5_0/lib/python3.9/site-packages/notebook/services/nbconvert
/opt/anaconda3/pkgs/nbconvert-6.4.4-py39hecd8cb5_0/info/test/nbconvert
/opt/anaconda3/pkgs/nbconvert-6.4.4-py39hecd8cb5_0/lib/python3.9/site-packages/nbconvert
/opt/anaconda3/pkgs/nbconvert-6.4.4-py39hecd8cb5_0/share/jupyter/nbconvert
/opt/anaconda3/lib/python3.9/site-packages/jupyter_server/tests/nbconvert
/opt/anaconda3/lib/python3.9/site-packages/jupyter_server/tests/services/nbconvert
/opt/anaconda3/lib/python3.9/site-packages/jupyter_server/nbconvert
/opt/anaconda3/lib/python3.9/site-packages/jupyter_server/services/nbconvert
/opt/anaconda3/lib/python3.9/site-packages/nbconvert
/opt/anaconda3/lib/python3.9/site-packages/notebook/nbconvert
/opt/anaconda3/lib/python3.9/site-packages/notebook/services/nbconvert
/opt/anaconda3/share/jupyter/nbconvert
\end{lstlisting}

\hypertarget{296a8c95}{}
\begin{lstlisting}[language=Python]
!find /Users/ -name "pandoc"
\end{lstlisting}

\begin{lstlisting}
/Users//nymath/Library/TeXShop/Engines/Inactive/pandoc
\end{lstlisting}

\hypertarget{87591abe}{}
\begin{lstlisting}[language=Python]
!which jupyter
\end{lstlisting}

\begin{lstlisting}
/opt/anaconda3/bin/jupyter
\end{lstlisting}

\hypertarget{291f956f}{}
\begin{lstlisting}[language=Python]
!which python # python解释器所在位置,的确是vscode中用到的那个。
\end{lstlisting}

\begin{lstlisting}
/opt/anaconda3/bin/python
\end{lstlisting}

\hypertarget{ffb615b4}{}
\begin{lstlisting}[language=Python]
\end{lstlisting}

\hypertarget{74eb307f}{}
\hypertarget{pandoc}{%
\subsection{Pandoc}\label{pandoc}}

\hypertarget{db8aa717}{}
\begin{lstlisting}[language=Python]
# 获取pandoc的安装路径
!find /Users -name "pandoc"
\end{lstlisting}

\begin{lstlisting}
/Users/nymath/Library/TeXShop/Engines/Inactive/pandoc
\end{lstlisting}

\hypertarget{9e039552}{}
\begin{lstlisting}[language=Python]
!open /Users/nymath/Library/TeXShop/Engines/Inactive/pandoc
\end{lstlisting}

\hypertarget{1fe0e645}{}
\begin{lstlisting}[language=Python]
# pandoc这个解释器?或者说函数?命令?保存的位置
!which pandoc
\end{lstlisting}

\begin{lstlisting}
/usr/local/bin/pandoc
\end{lstlisting}

\leavevmode\vadjust pre{\hypertarget{2208d79d}{}}%
开始pandoc的应用介绍了

\hypertarget{de037974}{}
\begin{lstlisting}[language=Python]
!pandoc --version
\end{lstlisting}

\begin{lstlisting}
pandoc 2.19.2
Compiled with pandoc-types 1.22.2.1, texmath 0.12.5.2, skylighting 0.13,
citeproc 0.8.0.1, ipynb 0.2, hslua 2.2.1
Scripting engine: Lua 5.4
User data directory: /Users/nymath/.local/share/pandoc
Copyright (C) 2006-2022 John MacFarlane. Web:  https://pandoc.org
This is free software; see the source for copying conditions. There is no
warranty, not even for merchantability or fitness for a particular purpose.
\end{lstlisting}

\hypertarget{6eb2a15a}{}
\begin{lstlisting}[language=Python]
!mkdir pandoc-test
\end{lstlisting}

\hypertarget{e3721cb3}{}
\begin{lstlisting}[language=Python]
!cd pandoc-test;\
pwd;\
\end{lstlisting}

\begin{lstlisting}
/Users/nymath/Desktop/python/python_quant/Statistical_learning/pandoc-test
\end{lstlisting}

\leavevmode\vadjust pre{\hypertarget{eea4fb86}{}}%
相关命令解读

\begin{lstlisting}
pandoc test1.md -f markdown -t html -s -o test1.html
\end{lstlisting}

\emph{test1.md} tells pandoc which file to convert.\\
\emph{-f} markdown -t html: from md to html.\\
\emph{-s} says to create a ``standalone'' file\\
\emph{-o} test1.html says to put the output in the file test1.html.

\hypertarget{e9f2141a}{}
\begin{lstlisting}
pandoc -f html -t markdown hello.html
\end{lstlisting}

将hello.html转化为hello.md

\hypertarget{e9d92533}{}
\begin{lstlisting}[language=Python]
!pandoc zsh_basics.ipynb -f ipynb -t latex -s -o zsh_basics.tex
\end{lstlisting}

\hypertarget{9438b4d2}{}
\begin{lstlisting}[language=Python]
zsh_in_jupyter.opendir()
\end{lstlisting}

\hypertarget{d987b502}{}
\begin{lstlisting}[language=Python]
\end{lstlisting}

\hypertarget{4e375e7c}{}
\begin{lstlisting}[language=Python]
\end{lstlisting}

\hypertarget{e4783d59}{}
\begin{lstlisting}[language=Python]
\end{lstlisting}

\hypertarget{dddb2f3c}{}
\begin{lstlisting}[language=Python]
\end{lstlisting}

\hypertarget{42c916b6}{}
\begin{lstlisting}[language=Python]
\end{lstlisting}

\hypertarget{506a98c7}{}
\begin{lstlisting}[language=Python]
\end{lstlisting}

\hypertarget{54253909}{}
\hypertarget{nbconvert}{%
\subsection{nbconvert}\label{nbconvert}}

\leavevmode\vadjust pre{\hypertarget{6228238f}{}}%
\href{https://nbconvert.readthedocs.io/en/latest/usage.html?highlight=pdf}{指南}\\
当然pandoc也非常重要,可以看看\\
\href{https://pandoc.org/}{Pandoc}

\hypertarget{d472e0de}{}
\hypertarget{html}{%
\subsubsection{HTML}\label{html}}

\hypertarget{840690bf}{}
\begin{lstlisting}[language=Python]
# HTML
!jupyter nbconvert --to html zsh_basics.ipynb
\end{lstlisting}

\begin{lstlisting}
[NbConvertApp] Converting notebook zsh_basics.ipynb to html
[NbConvertApp] Writing 595565 bytes to zsh_basics.html
\end{lstlisting}

\hypertarget{87d51d83}{}
\hypertarget{pdf}{%
\subsubsection{PDF}\label{pdf}}

PDF转化为pdf倒是挺费功夫的,不信的话直接转化看看,应该会报错,比如不支持中文,因此我们需要去弄一个template\\
这个文件所在路径为/opt/anaconda3/share/jupyter/nbconvert/templates/latex

\hypertarget{8c8f166b}{}
\begin{lstlisting}[language=Python]
!open /opt/anaconda3/share/jupyter/nbconvert/templates/latex 
\end{lstlisting}

\hypertarget{06137370}{}
\begin{lstlisting}[language=Python]
!jupyter nbconvert --to pdf zsh_basics.ipynb
\end{lstlisting}

\begin{lstlisting}
[NbConvertApp] Converting notebook zsh_basics.ipynb to pdf
[NbConvertApp] Writing 48048 bytes to zsh_basics.pdf
\end{lstlisting}

\hypertarget{62e24bad}{}
\begin{lstlisting}[language=Python]
!jupyter nbconvert --to pdf --template custom_template.tplx zsh_basics.ipynb
\end{lstlisting}

\hypertarget{4fb50372}{}
\begin{lstlisting}[language=Python]
\end{lstlisting}

\hypertarget{d9b2b500}{}
\hypertarget{python}{%
\subsubsection{Python}\label{python}}

\hypertarget{56272e70}{}
\begin{lstlisting}[language=Python]
!jupyter nbconvert --to python zsh_basics.ipynb
\end{lstlisting}

\begin{lstlisting}
[NbConvertApp] Converting notebook zsh_basics.ipynb to python
[NbConvertApp] Writing 4870 bytes to zsh_basics.py
\end{lstlisting}

\hypertarget{09bc2ee6}{}
\begin{lstlisting}[language=Python]
zsh()
\end{lstlisting}

\hypertarget{bad748bc}{}
\hypertarget{latex}{%
\subsubsection{Latex}\label{latex}}

\hypertarget{b9528858}{}
\begin{lstlisting}[language=Python]
!jupyter nbconvert --to latex zsh_basics.ipynb
\end{lstlisting}

\begin{lstlisting}
[NbConvertApp] Converting notebook zsh_basics.ipynb to latex
[NbConvertApp] Writing 31753 bytes to zsh_basics.tex
\end{lstlisting}

\leavevmode\vadjust pre{\hypertarget{35a6ca5c}{}}%
修改 documentclass{[}11pt{]}\{article\}\\
为\\
documentclass\{article\}\\
usepackage\{ctex\}

\hypertarget{1c7d164c}{}
\begin{lstlisting}[language=Python]
import zsh_in_jupyter
\end{lstlisting}

\hypertarget{6567a6f4}{}
\begin{lstlisting}[language=Python]
# 查看模组的说明
print(zsh_in_jupyter.__doc__)
\end{lstlisting}

\begin{lstlisting}

用于调用xxx

\end{lstlisting}

\hypertarget{8400b0b0}{}
\begin{lstlisting}[language=Python]
zsh_in_jupyter.opendir()
\end{lstlisting}

\hypertarget{26e90691}{}
\begin{lstlisting}[language=Python]
!open 
\end{lstlisting}

\hypertarget{2c6881f1}{}
\begin{lstlisting}[language=Python]
\end{lstlisting}

\hypertarget{5de6e2f3}{}
\begin{lstlisting}[language=Python]
\end{lstlisting}

\hypertarget{4ce8b5a7}{}
\hypertarget{work-with-files-and-directories}{%
\section{Work with Files and
Directories}\label{work-with-files-and-directories}}

似乎用的不太多,但还是写一点

\hypertarget{fee21310}{}
\hypertarget{create-a-file}{%
\subsection{Create a File}\label{create-a-file}}

Command \textbf{touch} is used for creating a file. (md
macos简直了,右键创建文件夹都得付费)

\hypertarget{fe2faaec}{}
\begin{lstlisting}[language=Python]
!touch test.sh
\end{lstlisting}

\hypertarget{22504561}{}
\begin{lstlisting}[language=Python]
zsh_in_jupyter.opendir()
\end{lstlisting}

\hypertarget{989345a2}{}
\hypertarget{create-a-directory}{%
\subsection{Create a Directory}\label{create-a-directory}}

\hypertarget{2cf3ca8f}{}
\begin{lstlisting}[language=Python]
\end{lstlisting}

\hypertarget{bc0b18fa}{}
\begin{lstlisting}[language=Python]
\end{lstlisting}

\hypertarget{bc02dd51}{}
\begin{lstlisting}[language=Python]
\end{lstlisting}

\hypertarget{d7e05175}{}
\begin{lstlisting}[language=Python]
\end{lstlisting}

\hypertarget{4665e0f1}{}
\begin{lstlisting}[language=Python]
\end{lstlisting}

\hypertarget{c414eefd}{}
\begin{lstlisting}[language=Python]
\end{lstlisting}

\hypertarget{a7292b00}{}
\begin{lstlisting}[language=Python]
\end{lstlisting}

\hypertarget{6f626e67}{}
\begin{lstlisting}[language=Python]
\end{lstlisting}

\hypertarget{71771c75}{}
\begin{lstlisting}[language=Python]
\end{lstlisting}

\hypertarget{87afa687}{}
\begin{lstlisting}[language=Python]
\end{lstlisting}

\hypertarget{50f27d9c}{}
\begin{lstlisting}[language=Python]
\end{lstlisting}

\hypertarget{800ab4ea}{}
\begin{lstlisting}[language=Python]
\end{lstlisting}


\end{document}
